\chapter{Related Works and Technologies} \label{ch:survey}

\section{Speech Feature Extraction}
\cite{1327109} investigate various feature extraction techniques - MFCC, LPCC, PLP for speech recognition on mobile phones. Their study shows that MFCC features give the most accurate result. In 
\cite{divya} a review of the latest feature extraction techniques is done. In this work, various advantages and disadvantages of techniques like RASTA and PLDA are pointed out. 

\section{Speech Modeling Techniques}
\cite{Mustafa2017} presents a comparative analysis of Hidden Markov Model, Multi-Layer Perceptron, and Dynamic Multi-Layer Perceptron. They conclude that DMLPs are infeasible on mobile phones, while HMMs are the most accurate, and MLPs are the fastest. With the increase in size of the vocabulary, the execution performance of HMM suffers. \cite{10.1007/978-981-10-3920-1_46} talks about implementing a lightweight speech recognition system using HMMs for Gujarati language and achieves 87\% accuracy.

\section{Distance Measures}
Cepstral coefficients are widely used in speech recognition. Various distortion methods have been applied to find the distance between two coefficients. \cite{1165058} brings out a novel distance measure, the Tohkura distance - a weighted distance that outperforms Itakura distance and Euclidean distance. It also mentions a more accurate Mahalanobis distance and the main difficulties for applying it to speech recognition.

\section{Native Development on Android}
\cite{5600170} discusses a new component development approach by using Java Native Interface technology. The Native Development Kit (NDK) is a set of tools that allows writing native code for Android in C++. Speech processing and recognition requires fine grain control over memory management and speed, hence NDK is preferred over Dalvik Virtual Machine. C and C++ libraries act as the glue code while the frontends can be written with Java or Swift.

\section{Network Speech Recognition}
In this mode of speech recognition, the feature extraction and recognition are both delegated to the network server. The server is capable of doing complex computation and returns extremely accurate results. One of the major constraints of this mode is the dependence on telephone network and the performance degradation caused by low bit rate codecs, which becomes severe in the presence of transmission errors \cite{Kumar_rethinkingspeech}.